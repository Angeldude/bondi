\documentclass{article}

\usepackage{hyperref}   % use for hypertext links, including those to external documents and URLs
\usepackage{verbatim}   % useful for program listings

\setlength{\parskip}{3pt plus 2pt}
\setlength{\parindent}{20pt}
\setlength{\oddsidemargin}{0.5cm}
\setlength{\evensidemargin}{0.5cm}
\setlength{\marginparsep}{0.75cm}
\setlength{\marginparwidth}{2.5cm}
\setlength{\marginparpush}{1.0cm}
\setlength{\textwidth}{150mm}

\begin{document}

\title{Bondi Language Definition}
\maketitle

\section{Introduction}
  TODO

\section{Keywords}

The following words are for bondi keywords and cannot be used as
identifiers.

\begin{verbatim}
  all                entry             let                spawn
  and                eqcons            lin                super
  as                 ext               match              then
  begin              extends           method             True
  class              False             new                type
  clone              for               newarray           Un
  datatype           fun               of                 view
  do                 if                rec                while
  else               in                ref                with
  end                lengthv           Ref
\end{verbatim}

\section{Literals}
\begin{verbatim}

  <Literal>:: = <IntegerLiteral>
              | <HexIntegerLiteral>
              | <FloatingPointerLiteral>
              | <BooleanLiteral>
              | <CharacterLiteral>
              | <StringLiteral>
              | <UnitLiteral>
  
  IntegerLiteral :: = [0-9]+    (* only nine digits is allowed in Bondi ) 
  Examples:

    Find boundary conditions for these values...

    1;; 
    0x10;;
    1.165;;

    True;;
    False;;  
    'b';;
    "bondi";;
    Un;;
\end{verbatim}

\section{Special Symbols}
\begin{verbatim}
  !            .
  $            /
  %            :
  &            ::
  &&           <
  *            =
  +=           ==
  -            >
  +            >>
  -->          ?
  ->           ^
  |            ||
\end{verbatim}

\section{Shell Actions}
\begin{verbatim}
  <shell_action>:: = %cd   <string> ( absolute path name )   
                   | %open <string> ( bondi file name    )
                   | %quit          ( end bondi session  )
                   | %hide <mode>
                   | %show <mode>
  mode  = infer 
        | declaration 
        | echo 
        | eval 
        | infer 
        | number 
        | parse 
        | prompt
        | specialise
        | types        

\end{verbatim}

\section{Operations}

\subsection{String}
\begin{verbatim}
  capitalizestring      copystring          indexfromstring   makestring
  concat                escapedstring       indexstring       rindexstring
  containsfromstring    float2string        int2string        substring
  containsstring        getcharstring       lengthstring      uppercasestring
\end{verbatim}

\subsection{Character Relational}
\begin{verbatim}
  greaterthanchar
  greaterthanorequalchar
  lessthanchar
  lessthanorequalchar
\end{verbatim}

\subsection{Float}
\begin{verbatim}
  acos      cos            floor         negatefloat     sinh
  asin      cosh           fmod          plusfloat       sqrt
  atan      dividefloat    log           pow             tan
  atan2     exponential    log10         randomfloat     tanh
  ceil      fabs           minusfloat    sin             timesfloat
\end{verbatim}

\subsection{Float Relational}
\begin{verbatim}
  greaterthanfloat
  greaterthanorequalfloat
  lessthanfloat
  lessthanorequalfloat
\end{verbatim}

\subsection{Integer}
\begin{verbatim}
  divideint
  minusint
  modint
  negateint
  plusint
  randomint
  timesint
\end{verbatim}

\subsection{Integer relational}
\begin{verbatim}
  greaterthanint
  greaterthanorequalint
  lessthanint
  lessthanorequalint
\end{verbatim}

\subsection{Socket}
\begin{verbatim}
  acceptclient           opentcp         
  close                  readall           
  gethost                readline          
  openfile               writeline         
  openserver             write           
\end{verbatim}


\begin{thebibliography}{5}
\end{thebibliography}
\end{document}
